% real_complex_DM_model.tex
\documentclass[11pt,a4paper]{article}
\usepackage[utf8]{inputenc}
\usepackage[T1]{fontenc}
\usepackage{amsmath,amssymb,amsfonts}
\usepackage{authblk}
\usepackage{graphicx}
\usepackage{hyperref}
\usepackage{siunitx}
\usepackage{braket}
\usepackage{geometry}
\usepackage{bm}
\usepackage{hyperref}
\geometry{margin=1in}
\hypersetup{colorlinks=true,linkcolor=blue,citecolor=blue,urlcolor=blue}

\title{A Complex-Field Two-State Model for Dark Matter: Theory, Dynamics and Cosmological Constraints}
\author{Kyrolles Ashraf Dawood}
\affil{Department of Physics, Faculty of Science, Assiut University, Asyut, Egypt\\Corresponding author: \texttt{kyrls.dawd1231@science.aun.edu.eg}}
\date{\today}

\begin{document}
\maketitle

\begin{abstract}
We present a minimal field-theoretic realization of a two-state dark-matter hypothesis using a single complex scalar field $\Phi(x)$. The real and imaginary components of $\Phi$ define two orthogonal sectors (denoted $R$ and $I$). A small explicit U(1)-breaking mass term induces coherent mixing and oscillations between the two components. We derive the effective two-state Hamiltonian, compute the transition probability $P_{R\to I}(t)$ in closed form, embed the dynamics into Boltzmann equations for cosmology, and map model parameters onto observational constraints (Planck, Lyman-$\alpha$, SIDM, Fermi-LAT, XENONnT). Full derivations, dimensional checks, and a table of physical constants (with references) are provided in appendices.
\end{abstract}

\tableofcontents

\section{Introduction}
\label{sec:intro}
(Summary) Motivation and overview. % keep short in manuscript body

\section{Model and notation}
\label{sec:model}
We employ natural units $\hbar=c=k_B=1$ in algebra; when giving numerical estimates we restore $\hbar$ explicitly and specify units (eV, GeV, s). Consider the complex scalar field $\Phi(x)$ with Lagrangian
\begin{equation}\label{eq:lag}
\mathcal{L} = \partial_\mu\Phi^\dagger\partial^\mu\Phi - m^2\Phi^\dagger\Phi - \left(\tfrac{1}{2}\varepsilon\,\Phi^2 + \mathrm{h.c.}\right) - V_{\rm int}(\Phi,{\rm SM}).
\end{equation}
Here $m$ (mass dimension 1) is the principal mass parameter, and $\varepsilon$ (mass$^2$) explicitly breaks the global U(1) symmetry and induces the real/imaginary mass splitting. For clarity we set $\varepsilon\in\mathbb{R}$ in the main text; general complex $\varepsilon$ is discussed in App.~\ref{app:complex_eps}.

Expand
\[
\Phi = \frac{1}{\sqrt2}(\phi_R + i\phi_I),
\]
with real fields $\phi_R,\phi_I$. The quadratic part gives masses
\[
m_R^2 = m^2 + \varepsilon,\qquad m_I^2 = m^2 - \varepsilon.
\]
Projecting to the single-particle nonrelativistic sector (localized states $\ket{R},\ket{I}$) yields an effective two-state Hamiltonian
\begin{equation}\label{eq:Hmat}
H = \begin{pmatrix} E_R & V \\[4pt] V & E_I \end{pmatrix},
\end{equation}
with average and half-difference
\begin{equation}\label{eq:defs}
E\equiv\frac{E_R+E_I}{2},\qquad \Delta\equiv\frac{E_R-E_I}{2},\qquad \Omega\equiv\sqrt{\Delta^2+V^2}.
\end{equation}
A parametric estimate (nonrelativistic projection, App.~\ref{app:projection}) gives
\begin{equation}\label{eq:Veps}
V \simeq \frac{\varepsilon}{2m}\,\mathcal{O}_{\rm wf},
\end{equation}
where $\mathcal{O}_{\rm wf}\leq 1$ is a dimensionless overlap factor (absorbing order-one geometry factors).

\section{Two-level quantum dynamics}
\label{sec:twolevel}
Solve $i\hbar\dot{\Psi}=H\Psi$ with initial $\Psi(0)=\ket{R}$. After factoring out global phase $e^{-iEt/\hbar}$ the solution gives the standard Rabi-like result (derivation in App.~\ref{app:diagonal}):
\begin{equation}\label{eq:Prtoi}
P_{R\to I}(t)\equiv |c_I(t)|^2 = \frac{V^2}{\Delta^2+V^2}\,\sin^2\!\left(\frac{\Omega t}{\hbar}\right).
\end{equation}
Dimensional consistency: $\Omega$ [energy], $\hbar$ [energy$\cdot$time] $\Rightarrow$ argument dimensionless.

Special limits:
\begin{itemize}
  \item Resonant ($\Delta\!=\!0$): $P=\sin^2(Vt/\hbar)$ (complete transfer possible).
  \item Off-resonant ($\Delta\gg V$): amplitude $\approx V^2/\Delta^2$.
\end{itemize}

If oscillations are fast relative to measurement or cosmological timescales, time-average gives
\begin{equation}\label{eq:avgP}
\langle P_{R\to I}\rangle_t = \frac{1}{2}\cdot\frac{V^2}{\Delta^2+V^2}.
\end{equation}

\section{Open-system effects: Lindblad and effective conversion rate}
\label{sec:lindblad}
In medium/decohering environments coherent oscillations are damped. The GKSL master equation
\begin{equation}\label{eq:gksl}
\dot{\rho} = -\frac{i}{\hbar}[H,\rho] + \sum_k\left(L_k\rho L_k^\dagger - \tfrac12\{L_k^\dagger L_k,\rho\}\right)
\end{equation}
is used. For dominant pure dephasing with jump operator $L=\sqrt{\gamma_\phi}\,\sigma_z$ (rate $\gamma_\phi$), adiabatic elimination of coherences (Appendix~\ref{app:lindblad}) yields the effective population-transfer rate
\begin{equation}\label{eq:GammaEff_main}
\Gamma^{\rm eff}_{R\to I} \;=\; \frac{V^2\,\gamma_\phi}{\Delta^2 + (\hbar\gamma_\phi)^2}.
\end{equation}
This expression (units $[{\rm time}]^{-1}$) interpolates between the coherent ($\gamma_\phi\to0$) and strongly-damped regimes.

\section{Cosmological embedding and Boltzmann equation}
\label{sec:cosmo}
Define number density $n$ and comoving yield $Y=n/s$ with entropy density $s$. The Boltzmann equation for $Y(x)$ with $x\equiv m/T$ is
\begin{equation}\label{eq:Boltzmann_main}
\frac{dY}{dx} = -\frac{s\langle\sigma v\rangle}{xH}\big(Y^2 - Y_{\rm eq}^2\big) - \frac{\Gamma_{\rm decay}}{xH}Y + \frac{S_{\rm fi}(x)}{xH}.
\end{equation}
Given $Y_\infty$,
\begin{equation}\label{eq:Omega_conv}
\Omega_c h^2 = \frac{m\,s_0\,Y_\infty}{\rho_{\rm crit}/h^2} \simeq 2.744\times10^8 \left(\frac{m}{\rm GeV}\right) Y_\infty,
\end{equation}
where the numerical prefactor uses constants in Appendix~\ref{app:constants} (see references).

If oscillations are rapid and $I$ gravitates, total relic abundance is set by $Y_\infty$ regardless of partition $P_R,P_I$; conversely slowly-varying conversions change the effective observable distribution in $R$ vs $I$ over cosmological time.

\section{Observational constraints and parameter mapping}
\label{sec:constraints}
We confront the model to key empirical bounds:
\begin{itemize}
  \item \textbf{Planck 2018:} $\Omega_c h^2 = 0.120\pm0.001$. \cite{Planck2018params}
  \item \textbf{Lyman-$\alpha$:} lower bounds on thermal WDM $\gtrsim$ few keV (Viel et al. 2013). \cite{Viel2013}
  \item \textbf{SIDM / Bullet Cluster:} $\sigma/m\lesssim 0.1$--$1\ \mathrm{cm^2/g}$. \cite{Clowe2006bullet,TulinYu2018}
  \item \textbf{Indirect detection (Fermi-LAT):} limits on annihilation/decay lifetimes (e.g., Ackermann et al. 2015). \cite{Ackermann2015}
  \item \textbf{Direct detection (XENONnT):} upper limits on $\sigma_{\rm SI}$ (first results 2023). \cite{XENONnT2023}
\end{itemize}
Mapping procedures (implemented in provided code): solve Eq.~\eqref{eq:Boltzmann_main} on parameter grid $(m,\langle\sigma v\rangle,\Gamma,\;V,\gamma_\phi)$ and extract contours where $\Omega_ch^2=0.12$; then apply observational exclusions to mask forbidden regions.

\section{Illustrative numeric examples}
\label{sec:examples}
Restore $\hbar=6.582119569\times10^{-16}\ \mathrm{eV\cdot s}$ for numerics. For resonant $\Delta=0$,
\[
T_{\rm P}=\frac{\pi\hbar}{V}.
\]
Example:
\[
V=10^{-4}\ \mathrm{eV} \quad\Rightarrow\quad T_{\rm P}\approx\frac{\pi\times6.5821\times10^{-16}}{10^{-4}}\ \mathrm{s}\approx 2.07\times10^{-11}\ \mathrm{s}.
\]
To get $T_{\rm P}\sim t_U\approx 4.35\times10^{17}\ \mathrm{s}$ needs $V\sim1.5\times10^{-33}\ \mathrm{eV}$.

\section{Discussion and conclusions}
\label{sec:conclusion}
 Advantages, caveats, and paths forward for more detailed numerical/observational confrontation.

\appendix
\section{Diagonalization of the two-state Hamiltonian}
\label{app:diagonal}
We present here the full, step-by-step diagonalisation of the Hamiltonian \eqref{eq:Hmat} and solution for time evolution.

Write $H=E\mathbb{I} + H'$ with
\[
H'=\begin{pmatrix}\Delta & V\\ V & -\Delta\end{pmatrix},\qquad
\Delta=\tfrac{1}{2}(E_R-E_I).
\]
Eigenvalues of $H'$ are $\pm\Omega$ with $\Omega=\sqrt{\Delta^2+V^2}$. Thus eigenvalues of $H$ are $E\pm\Omega$.

Normalized eigenvectors for $H'$ corresponding to $+\Omega$ and $-\Omega$ may be taken as
\[
\ket{+} = \begin{pmatrix}\cos\theta\\ \sin\theta\end{pmatrix},\qquad
\ket{-} = \begin{pmatrix}-\sin\theta\\ \cos\theta\end{pmatrix},
\]
with mixing angle $\theta$ defined by $\tan(2\theta)=V/\Delta$ (equivalently $\sin2\theta=V/\Omega$, $\cos2\theta=\Delta/\Omega$). Starting from initial $\ket{R}=(1,0)^T$, expand in eigenbasis and propagate phases $e^{-i(E\pm\Omega)t/\hbar}$. After algebra one finds (removing global $e^{-iEt/\hbar}$) the amplitude in state $\ket{I}$:
\[
c_I(t) = \frac{V}{\Omega}\sin\!\left(\frac{\Omega t}{\hbar}\right)\,,
\]
hence $P_{R\to I}=|c_I|^2$ yields Eq.~\eqref{eq:Prtoi}. (All algebraic steps are straightforward; check normalization at $t=0$ and small-$t$ expansion.)

\section{Derivation of the effective rate from a Lindblad model}
\label{app:lindblad}
Consider $H'= \Delta\sigma_z + V\sigma_x$ (energies) and Lindblad pure dephasing operator $L=\sqrt{\gamma_\phi}\,\sigma_z$. GKSL equation for the density matrix elements (in basis $\{\ket{R},\ket{I}\}$) gives (writing $\rho_{ij}$)
\begin{align*}
\dot{\rho}_{11} &= -\frac{i}{\hbar}V(\rho_{12}-\rho_{21}),\\
\dot{\rho}_{12} &= -\frac{i}{\hbar}\big(2\Delta\rho_{12} + V(\rho_{11}-\rho_{22})\big) - 2\gamma_\phi \rho_{12}.
\end{align*}
Assume coherence $\rho_{12}$ relaxes fast (adiabatic elimination): set $\dot{\rho}_{12}\approx0$ and solve for $\rho_{12}$:
\[
\rho_{12} \simeq \frac{i(V/\hbar)(\rho_{11}-\rho_{22})}{i(2\Delta/\hbar) + 2\gamma_\phi}
= \frac{(V/\hbar)(\rho_{11}-\rho_{22})}{(2\Delta/\hbar) - i2\gamma_\phi}.
\]
Compute $\dot{\rho}_{11} = (2V/\hbar)\mathrm{Im}(\rho_{12})$; carrying out algebra yields
\[
\dot{\rho}_{11} = -\Gamma^{\rm eff}_{R\to I}\,\rho_{11} + \Gamma^{\rm eff}_{I\to R}\,\rho_{22},
\]
with (for approximately symmetric case)
\begin{equation}\label{eq:GammaEff_app}
\Gamma^{\rm eff}_{R\to I} \;=\; \frac{V^2\,\gamma_\phi}{\Delta^2 + (\hbar\gamma_\phi)^2},
\end{equation}
matching Eq.~\eqref{eq:GammaEff_main}. The derivation above (explicit algebra) ensures units: numerator $[E^2]\cdot[1/t]$, denominator $[E^2]$ $\Rightarrow$ $[1/t]$.

\section{Projection from field Lagrangian to two-state Hamiltonian}
\label{app:projection}
Sketch: start from quadratic part of \eqref{eq:lag}, expand fields in single-particle wavefunctions $\Phi(\mathbf{x},t)\approx\psi_R(\mathbf{x})a_R(t)+\psi_I(\mathbf{x})a_I(t)$, integrate spatial coordinates and perform nonrelativistic reduction. Off-diagonal term arises from $\varepsilon\Phi^2$ bilinear leading to estimate \eqref{eq:Veps}; factors of order unity are absorbed in $\mathcal{O}_{\rm wf}$.

\section{Table of constants and numeric values}
\label{app:constants}
We collect the numerical constants used (values and sources). Where needed we restore SI units; conversions given in the text.

\begin{table}[h!]
\centering
\begin{tabular}{llc}
\hline
Quantity & Value & Reference \\
\hline
Reduced Planck constant $\hbar$ & $6.582119569\times10^{-16}\ \mathrm{eV\cdot s}$ & CODATA 2018 \cite{CODATA2018} \\
Today's entropy density $s_0$ & $2891.2\ \mathrm{cm^{-3}}$ (approx.) & Planck 2018 / cosmology references \cite{Planck2018params} \\
Critical density per $h^2$ & $\rho_{\rm crit}/h^2 \approx 1.05375\times10^{-5}\ \mathrm{GeV\,cm^{-3}}$ & standard cosmology (Planck) \cite{Planck2018params} \\
Age of Universe & $t_U \approx 4.35\times10^{17}\ \mathrm{s}$ & Planck 2018 cosmology \cite{Planck2018params} \\
\hline
\end{tabular}
\caption{Physical constants and cosmological reference values used in numeric conversions.}
\end{table}

\section{License}

The text of this work (manuscript, documentation and README) is licensed under the Creative Commons Attribution 4.0 International (CC BY 4.0) license.  
The code in this repository is released under the MIT License. The full license texts are included in the project repository under the file \texttt{LICENSE}.

Online references:
\begin{itemize}
  \item Repository: \url{https://github.com/scikyrollesashraf/complex-dm-two-state}
  \item Citable archived release (Zenodo DOI): \href{https://doi.org/10.5281/zenodo.17772760}{10.5281/zenodo.17772760}
  \item Creative Commons (CC BY 4.0) legal code: \url{https://creativecommons.org/licenses/by/4.0/}
  \item MIT License (text contained in the repository's LICENSE file).
\end{itemize}

% Bibliography
\bibliographystyle{unsrt}
\bibliography{references}

\end{document}
